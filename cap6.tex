%%%%%%%%%%%%%%%%%%%%%%%%%%%%%%%%%%%%%%%%%%%%%%%%%%%%%%%%%%%%%%%%%%%%%%%%%%%%%
% Chapter 6: Summary and Conlusions
%%%%%%%%%%%%%%%%%%%%%%%%%%%%%%%%%%%%%%%%%%%%%%%%%%%%%%%%%%%%%%%%%%%%%%%%%%%%%%%

%++++++++++++++++++++++++++++++++++++++++++++++++++++++++++++++++++++++++++++++

CodeLab was born as a tool that aims to extend the functionality
of other tools such as Github Classroom, adding specific functions
for the teachers to support the management of courses and 
the correction of programming labs.

The platform has been designed with ease of use in mind for those
who are not familiar with GitHub. On the other hand, emphasis has
been placed on fulfilling the Free Software standards as much as
possible,
facilitating collaborative development and the inclusion of new features by other programmers.

We believe that the facilities provided by  CodeLab are
useful for lecturers who have large groups of students
or who are not familiar with the Github environment.

Version control offers many advantages to developers.
Currently, most development companies use the git version control system, 
and consequently it is essential that students learn to handle git correctly and
learn teamwork techniques as well as all the tools that Github
provides such as issues or projects.

In the future, I would like to continue developing CodeLab, improving
it and adding new functionalities. One of the first improvements
that is proposed is the use of a front-end library such as Vue or
React to improve the visual quality of the web platform. A new
functionality that I would like to add is the possibility of more
than one teacher per classroom.
